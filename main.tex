\documentclass[12pt,oneside]{book}

\usepackage[a4paper, total={6.1in, 10in}]{geometry} % margin stuff
\usepackage{amsmath}
\usepackage{graphicx}
\usepackage{varwidth}
\usepackage{tcolorbox}
\usepackage{hyperref}
\usepackage{changepage}
\usepackage{python}
\usepackage{tabularx}


\hypersetup{
    colorlinks=true,    
    urlcolor=blue,
}

\begin{document}

%%%Tital Page%%%
\begin{center}
	\vspace*{\baselineskip} % White space at the top of the page
	
	\rule{\textwidth}{1.6pt}\vspace*{-\baselineskip}\vspace*{2.5pt}
	\rule{\textwidth}{0.4pt}
	
	\vspace{0.75\baselineskip}
	
	{\huge Mathematics 104}\\
	Statistics\\ 
	\vspace{0.25\baselineskip}
	
	\rule{\textwidth}{0.4pt}\vspace*{-\baselineskip}\vspace{3.7pt} 
	\rule{\textwidth}{1.6pt} 
\end{center}
\pagestyle{empty}



%%%Data%%%
\section*{Data}
\data


\section*{Project Questions}
\hspace*{.02\textwidth}%
\begin{minipage}{.96\textwidth}
\subsection*{1. Calculate the sample mean.}
The sample mean is $\xbar$.

\subsection*{2. Calculate the sample standard deviation.}
The sample standard deviation is $\sd$.

\subsection*{3. Plot a histogram and Q-Q plot of your data against the normal distribution. Comment
on whether the data appears to be normally distributed or not (2 sentences max).}
\begin{center}
\includegraphics[width=.4\textwidth]{hist.png}
\includegraphics[width=.4\textwidth]{qq.png}
\end{center}
The data looks relatively normally distributed however has an outlier at the positive side.
\end{minipage}%
\newpage
\begin{minipage}{.96\textwidth}
\subsection*{4. Discuss if assumptions required for the sample mean to be nearly normal are satisfied
(3 sentences max).}
The data set I've been give is ``randomly sampled" so the data will be independent.
The sample size is 40 which is greater than 30 so therefore the sample is large.
\ifdim \sk pt<0.5pt {
	\ifdim \sk pt>-0.5pt {
		From the charts above, the data does not seem to have a strong skew.
	} \else {
		From the charts above, the data seems to have a \ifdim \sk pt>-1.0pt {slight} \fi negative skew.
		
	} \fi
} \else {
From the charts above, the data seems to have a \ifdim \sk pt<1.0pt {slight} positive skew.
} \fi
\subsection*{5. Assuming the assumptions are satisfied, calculate the standard error of the mean.}
The standard error of the mean is $\se$.
\subsection*{6. Calculate a 95\% confidence interval for the average year-to-date stock return of S\&P
500 companies.}
The 95\% confidence interval is $\intervalA$.
\subsection*{7. Calculate a 99\% confidence interval for the average year-to-date stock return of S\&P
500 companies.}
The 99\% confidence interval is $\intervalB$.

\subsection*{8. Perform a hypothesis test which tests if the average year-to-date stock return of S\&P
500 companies is equal to zero. Clearly setup your hypothesis test and report a p-value (to 3 significant figures).}
$P=$``Stock Price Returns"\\
$H_0:\mu_P=0$\\
$H_A:\mu_P\neq0$\\
The p-value for the mean of $\xbar$ is $\pvalue$.
\subsection*{9. Would you reject the null hypothesis using a significance level of $\boldsymbol{\alpha}$ = 0.05?}
\ifdim \pvalue pt>0.05pt {
I would not reject the null hypothesis at $\alpha = 0.05$ as $p=\pvalue>0.05$. Also $0 \in \intervalA$.
}\else {
I would reject the null hypothesis at $\alpha = 0.05$ as $p=\pvalue<0.05$. Also $0 \not\in \intervalA$.
} \fi
\subsection*{10. Would you reject the null hypothesis using a significance level of $\boldsymbol{\alpha}$ = 0.01?}
\ifdim \pvalue pt>0.01pt {
I would not reject the null hypothesis at $\alpha = 0.01$ as $p=\pvalue>0.01$. Also $0 \in \intervalB$.
} \else {
I would reject the null hypothesis at $\alpha = 0.01$ as $p=\pvalue<0.01$. Also $0 \not\in \intervalB$.
} \fi
\end{minipage}%

\begin{tcolorbox}
{\large Source Code}\\
\url{https://github.com/flippers2652/104Project}
\end{tcolorbox}
\end{document}

"""
